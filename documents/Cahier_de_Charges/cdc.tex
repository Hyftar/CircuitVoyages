\documentclass{scrreprt}
\usepackage{listings}
\usepackage{underscore}
\usepackage{graphicx}
\usepackage[bookmarks=true]{hyperref}
\usepackage[utf8]{inputenc}
\usepackage[french]{babel}
\renewcommand{\contentsname}{Table des matières}
\def\myversion{1.0 }
\def\projectname{[NOM DU PROJET]}
\hypersetup{
    bookmarks=false,    % show bookmarks bar?
    pdftitle={Cahier de charges - \projectname},    % title
    pdfauthor={Simon Landry},                       % author
    pdfsubject={Cahier de charges du site circuit voyages},  % subject of the document
    pdfkeywords={},        % list of keywords
    colorlinks=true,       % false: boxed links; true: colored links
    linkcolor=black,       % color of internal links
    citecolor=black,       % color of links to bibliography
    filecolor=black,       % color of file links
    urlcolor=blue,         % color of external links
    linktoc=page           % only page is linked
}

\usepackage{hyperref}
\begin{document}
\pagenumbering{gobble}
\begin{flushright}
    \rule{16cm}{5pt}\vskip1cm
    \begin{bfseries}
        \Huge{CAHIER DE \\ CHARGES}\\
        \vspace{1.5cm}
        pour\\
        \vspace{1.5cm}
        \projectname\\
        \vspace{1.5cm}
        \LARGE{Version \myversion}\\
        \vspace{1.5cm}
        Préparé par :\\
        Nicholas Massé\\
        David Godin\\
        Philippe Bourassa\\
        Simon Landry\\
        et Keven Aubin\\
        \vspace{1.5cm}
        \today\\
    \end{bfseries}
\end{flushright}
\newpage
\pagenumbering{roman}
\tableofcontents
\newpage
\pagenumbering{arabic}
\chapter{Présentation du projet}

\projectname est un projet destiné à mettre en place une application
web afin d'offrir des services similaires à ceux d'une agence de voyage.

Ciblant une clientèle un peu plus précise, l'entreprise a pour but
d'offrir des voyages hors normes en veillant à ce que toutes les
demandes et volontés du client soient comblées. Les offres de voyage
ont un air standardisé mais offrent un tel éventail d'options qu'elles
font plutôt partie d'une démarche «sur mesure».

Le principe de la compagnie est de développer et d'organiser des
voyages selon des thématiques spécifiques, appelées des «circuits». Les
circuits sont destinés à une clientèle internationale et se veulent
accessibles de n'importe où. Le client ayant sélectionné sa thématique
de préférence, il pourra construire son voyage en choisissant son
propre point de départ et point d'arrivée à l'intérieur du circuit
proposé.

\chapter{Objectifs et environnements}
\section{Objectifs}
Se voulant être un projet d'envergure internationale, la mise en place
de l'application \projectname doit passer par certaines étapes.

Tout d'abord, \projectname a l'intention de se démarquer de ses
concurrents par son offre de service approfondie, et devra être adaptée
à tous les types de clientèle. Malgré tout, ses services assez onéreux
forcent la compagnie à s'adresser à une clientèle plus âgée et devra
donc axer ses services dans cette direction.

\projectname a la volonté de devenir un leader dans les offres de
voyages de luxe, et dans cette mesure s'attend à avoir une
infrastructure technologique appropriée.


\section{Environnements}
La mise en place de l'application web inclut trois branches
principales: les environnements «Client», «Admin» et «Personnel».

L'environnement «Client» permettra à n'importe quel internaute de
naviguer sur le site et de prendre connaissance des circuits de voyage
offerts par la compagnie. Le client aura également la possibilité de
créer son compte pour avoir accès à l'inscription sur les circuits, à
des informations personnalisées, à son propre profil ainsi qu'à son
historique de voyage.

L'environnement «Admin» permettra à l'administrateur ou son mandataire
de modifier les offres de circuit, de gérer les comptes clients ainsi
que les promotions et les rabais

L'environnement «Personnel» donnera accès à la gestion des envois de
promotions et rabais. Il permettra aussi la consultation et la gestion
des paiements par les clients.

\chapter{Ressources allouées}
\section{Financement et matériel}

\projectname ne possède pas de limite budgétaire et c'est dans cette
optique que la qualité de l'application web doit être inégalée. Bien
qu'une limite temporelle et technologique existe, l'entreprise a
confiance en l'équipe de développement en ce qui a trait à la gestion
des ressources matérielles.

\section{Équipe de développement}

La compagnie fait appel à une équipe de jeunes développeurs composée de
finissants du niveau collégial québécois. Leurs origines
professionnelles et personnelles sont diversifiées et en font un atout
principal.

Ci-dessous se trouve un petit descriptif de chacun des membres de
l'équipe de développement.

\subsection{Keven Aubin}

Keven Aubin est un développeur web terminant sa formation au Collège de
Maisonneuve. Son expérience en développement front-end et back-end en
feront un membre polyvalent de l'équipe.
Expérimenté à la tâche de Scrum Master, Keven saura mener l'équipe dans
ce rôle lors des scrums journaliers et saura protéger l'équipe des
obstacles et distractions qui l'entourent.
Titulaire d'un diplôme de droit notarial, ce développeur utilise son
expérience de la communication pour mieux communiquer avec le client et
ainsi comprendre plus facilement les besoins et désirs du maître
d'ouvrage.

\subsection{Philippe Bourassa}

[...]

\subsection{Simon Landry}

Simon Landry est un développeur back-end terminant sa formation au Collège de Maisonneuve.
Son expertise en développement sur divers projets «Open Source», qui s'échelonne sur plus d'une dizaine d'années,
lui permettera de conseiller ses collègues sur les meilleures pratiques de développement et les multiple outils qui entourent le développement.
Après une année au sein des Forces Armées Canadiennes en tant qu'officier, ce développeur saura utiliser ses compétences en leadership pour mener le projet à termes.

\subsection{David Godin}

[...]

\subsection{Nicholas Massé}

[...]


\end{document}
